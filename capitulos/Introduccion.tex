\chapter{Introducción}


\section{El marketing digital en la web}


\subsection{¿Que és?}

El marketing digital son técnicas y estrategias de comercialización usando medios digitales, tales como dispositivos móviles, televisiores digitales y ordenadores.

\vspace{5 mm}

\textbf{¿Cuáles son las principales diferencias respecto al marketing tradicional?} 

\vspace{5 mm}

\begin{itemize}

\item \textbf{Personalización}: El marketing digital pretende obtener información del usuario más personalizada. Por ello aplica técnicas que permiten que se le sugieran a los internautas información de aquello que está interesado, basado en búsquedas previas o en sus preferencias definidas.

\item \textbf{Masivo}: Se obtener un gran número de usuarios que formen parte de tu público objetivo invirtiendo muchos menos dinero que en el marketing tradicional.


\end{itemize}

\subsection{Evolución}

La concepto de marketing digital en la web durante sus inicios hasta la actualidad ha ido ido variando. En los años noventa con la aparición de los primeros banners, aparecen las primeras técnicas de marketing en páginas web aunque este primer concepto que se tiene es mucho más básico y se basaba principalmente en hacer publicidad hacia los usuarios para captar posibles clientes.

\vspace{5 mm}


Con la llegada de las redes sociales y la aparición de los smartphones el concepto de marketing cambia, no se basa únicamente en promocionar un producto. Se pretende crear una estrategia de venta basada en la perspectiva del cliente. Reconociendo e investigando en las areas que ayuden a los clientes a pensar que sus opiniones importan, genera una fidelidad hacia la marca. Aquí es donde las redes sociales juegan un gran papel, ya que permiten compartir todo tipo de contenido(videos,enlaces,textos) que se asocian a nuestras opiniones y gustos personales. Todo este contenido permite a las empresas conocer más detalladamente a un potencial cliente y así crear estrategías de marketing para fidelizar con él.


\subsection{Herramientas}

\vspace{5 mm}

\textbf{La publicidad web} 

\vspace{5 mm}


\vspace{5 mm}

\textbf{La analítica web} 

\vspace{5 mm}


\vspace{5 mm}

\textbf{La redes sociales} 

\vspace{5 mm}

\subsection{El caso de Twitter}