\chapter{Objetivos}


\section{Objetivo Principal}

\displayquote{``Desarrollar una herramienta que permita conocer, según localización geográfica, lo que los usuarios están publicando en distintas redes sociales acerca de temas culinarios orientada para su uso en el marketing y promoción de un portal de cocina.''}

Ello permitirá enfocar mejor las campañas de difusión y promoción de productos de ese sector a un público concreto al localizarlo geográficamente.

\vspace{5 mm}

\section{Objetivos Específicos}

\begin{itemize}
  \item Definir los requisitos y espicificaciones de la aplicación.
  \item Estudiar de las diversas API's a utilizar en el proyecto y conocer su uso.
  \item Diseñar una arquitectura de software capaz de recoger información de las diferentes redes sociales y almacenarla de forma eficaz en la base de datos.
  \item \textbf{Testear y Probar} la aplicación en diferentes navegadores y dispositivos.
  \item Proporcionar un \textbf{producto minimo viable(PMV)} de la aplicación y un modelo de negocio basado en el.
\end{itemize}

\subsection{Lean Canvas}

Para tener detallados todos los objetivos reales de la aplicación, se lleva a cabo la metodología Lean Canvas, que permite reflejar los problemas y detallar las acciones necesarias para detallar el producto. Los distintos elementos del Lean Canvas:

\begin{itemize}
  \item Segmento de Clientes: El público al que va enfocado. Son usuarios amantes de la gastronomía y la cocina en general. Cualquier tipo de persona interesada.
  \item Problemas: Actualmente existen muchos blogs de cocina, que enseñan técnicas y recetas para mejorar tus dotes culinarias y conocer platos típicos regionales.
  \item Propuesta de Valor:
  \item Solución:
  \item Canales:
  \item Fuentes de Ingreso:
  \item Estructura de Costes:
  \item Métricas Clave:
  \item Ventaja Injusta:
\end{itemize}
