\chapter{Metodología}

Para la realización de este proyecto, se estimó una duración de 5 meses donde 1 mes se dedicaría a la investigación, 1 mes para la especificación y diseño y 3 meses para el desarrollo. 

\vspace{5 mm}

Para elegir el marco de trabajo a usar en el desarrollo, se tienen en cuenta algunas de las características de las aplicaciones web:

\begin{description}

\item [Escalabilidad] La habilidad para manejar el crecimiento continuo del trabajo de manera fluida.  

\item [Interoperabilidad] La facilidad para comunicarse con diferentes protocolos e interfaces de datos. 

\item [Capacidad de Prueba] La habilidad para medir el correcto funcionamiento del sistema y sus componentes mediante pruebas. 

\end{description}


\vspace{5 mm}

Con esas características y sabiendo que el desarrollo del sistema se creará de forma iterativa, se opta por utilizar una metodología ágil. Despues de estudiar detalladamente cada una de las metodologías ágiles, finalmente se optó por usar SCRUM \cite{scrum}.

\vspace{5 mm}

La metodología SCRUM se basa en la realización de pequeños sprints(periodos en el cual se lleva a cabo el trabajo) de 2 o 3 semanas. Al inicio de cada sprint se lleva a cabo una reunión de planificación donde se especifica el trabajo que se va a realizar y se estiman el tiempo en horas que se va a tardar. Cuando finalize el sprint se realizará una revisión comprobando que se ha completado el trabajo.

\vspace{5 mm}

Ventajas de trabajar con la metodología SCRUM:

\begin{itemize}

\item Reducción de riesgos. Como el alcance está limitado al entregable del sprint comprometido, la aparición de riesgos se limita únicamente a lo que se va a desarrollar.

\item Se pueden priorizar los requisitos de la aplicación por valor y coste. En función al valor que aportan que aportan a la aplicación y el coste que supone desarrollarlas, se priorizan para proporcionar el resultado más óptimo en el proyecto. Esta lista de requisitos priorizada se denomina ``Product Backlog''.

\item Flexibilidad y adaptación. Al final de cada sprint se puede aprovechar la parte completada para hacer pruebas y sobre el resultado obtenido tomar decisiones.

\item Productividad y calidad. SCRUM se sirve de un proceso de mejora continua, comunicación diaria entre las partes implicadas, estimaciones de esfuerzos conjuntas y entrega de productos de forma regular. Todo ello, con la consigna de mejorar y simplificar la iteración anterior.

\end{itemize}