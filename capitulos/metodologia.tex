\chapter{Metodología}

Para la realización de este proyecto, se estimó una duración de 5 meses donde 1 mes se dedicaría a la investigación, 1 mes para la especificación y diseño y 3 meses para el desarrollo. 

\vspace{5 mm}

Para elegir el marco de trabajo a usar en el desarrollo se tienen en cuenta algunas de las características de las aplicaciones web:

\begin{itemize}

\item La habilidad para manejar el crecimiento continuo del trabajo de manera fluida(Escalabilidad).  

\item La facilidad para comunicarse con diferentes protocolos e interfaces de datos(Interoperabilidad). 

\item La habilidad para medir el correcto funcionamiento del sistema y sus componentes mediante pruebas(Capacidad de Prueba). 

\end{itemize}


\vspace{5 mm}

 Con esas características y sabiendo que el desarrollo del sistema se creará de forma iterativa, se opta por utilizar una metodología ágil. Despues de estudiar detalladamente cada una de las metodologías ágiles, finalmente se optó por usar Scrum.

\vspace{5 mm}

La metodología Scrum se basa en la realización de pequeños sprints(periodos en el cual se lleva a cabo el trabajo) de 2 o 3 semanas. Al inicio de cada sprint se lleva a cabo una reunión de planificación donde se especifica el trabajo que se va a realizar y se estiman el tiempo en horas que se va a tardar. Cuando finalize el sprint se realizará una revisión comprobando que se ha completado el trabajo.


