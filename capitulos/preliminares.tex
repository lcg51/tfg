\chapter*{Preámbulo}

Desde mis inicios en el grado de Ingeniería Multimedia, siempre me ha despertado mucho interés todo lo relacionado con el desarrollo web y
con la gestión de contenidos. En otros campos, la programación se volvía monótona y aburrida, mientras que con la programación web crecía mi interés y
las ganas de aprender, debido a la manera tan gráfica y sencilla como se representa la información. Esto, más el factor extra de mi interés por
el diseño web, hicieron que empezara a desarrollar mis primeros proyectos web.

\vspace{5 mm}

Conforme han ido pasando mis años de estudio, he ido conociendo nuevas herramientas para desarrollo, nuevos lenguajes de programación web y nuevas técnicas que me
han ido proporcionado una base sólida. Fue tal mi interés por el tema web que, durante la realización de mis prácticas de empresa, aprendí a aplicar al mismo algunas técnicas basicas de
marketing digital. Aunque sin profundizar mucho en cada una, había aprendido las bases de cada una de las fases del desarrollo web integral.

\vspace{5 mm}

Entonces llegó el momento de presentar mi propuesta de Trabajo de Grado. Fue entonces cuando, el que luego sería mi tutor: el profesor Pedro Pernías, me propuso una idea de proyecto interesante que yo decidí
desarrollar, aplicando los conocimientos que ya tenía y englobando todas las fases de un proyecto web: planificación,arquitectura,diseño,desarrollo y marketing.

\vspace{5 mm}

Durante este trabajo aplicaré asignaturas como Usabilidad y Acesibilidad, Negocio Multimedia, Análisis y Especificación de Requisitos, Sistemas Multimedia,
Programación Hipermedia 1 y 2, Sistemas Multimedia Avanzados.




\chapter*{Agradecimientos}

Me gustaría empezar dando las gracias a mi tutor Pedro Pernías. Por sus ideas y consejos que me ha proporcionado durante todo este aprendizaje.
Y por su entusiasmo y pasión que me ha transmitido en todas sus clases. Sin él este trabajo no habría cobrado vida.

\vspace{5 mm}

En segundo lugar, agradecer también a mis padres Luis y Raquel, todo el apoyo e interés que me han demostrado durante el desarrollo de este proyecto.
Y agradecer todos sus consejos que me han proporcionado durante estos años de carrera.

\vspace{5 mm}

También quiero dar las gracias a la empresa 3dids.com empresa donde he trabajado durante 7 meses, por todo lo que me ha enseñado, ya que sin esos
conocimientos no podría haber desarrollado el proyecto con tanta fluidez.


\vspace{5 mm}

Por último agradecer a mi compañero de clase Pablo Pernías, por sus consejos de desarrollo y sugerencias de ampliación y mejora de la
 aplicación que han servido de gran ayuda para mejorar el proyecto.
